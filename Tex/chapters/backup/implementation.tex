
\section{Section 1}
\label{sec:impl_1}

Lets begin with an equation
\begin{equation}
\label{eq:forward_chained_rule}
P(z_0,z_k) \leftarrow Q_1(z_0,z_1), Q_2(z_1,z_2), \dots, Q_k(z_{k-1},z_k), R_1(x_1), \dots,  R_n(x_n),
\end{equation}
where all $x_1, \dots, x_n \in \{z_0, \dots, z_k\}$.

Also a code snippet
\begin{minted}{prolog}
invented(1..num_inv).
:- #count { ID,P1,P2,P3 : meta(ID,P1,P2,P3) } != size.
\end{minted}
where \verb|meta/4| represents a metasubstitution.

\section{Encodings}
\subsection{Encoding 1}
\subimport{implementation/}{encoding_1.tex}
\subsection{Encoding 2}
\subimport{implementation/}{encoding_1.tex}


For you to understand the encoding, you have to look at this code:
\begin{figure}
\begin{minted}[linenos,numberblanklines=false]{prolog}
state(X) :- pos_ex(_,X,_).
state(X) :- neg_ex(_,X,_).
state(Y) :- d(_,_,Y).

unary_bg(P,X) :- P = @unary(X), state(X).
binary_bg(P,X,Y) :- (P,Y) = @binary(X), state(X).
d(P,X,Y) :- binary_bg(P,X,Y).
\end{minted}
\caption{Import of background knowledge in \encoding{Cli}{FC}}
\label{fig:bk_import_fc}
\end{figure}
Note that \verb|unary_bg/2| is in the code.

If you still havent understood this yet, look at the following example \Cref{ex:fc_bk}:
\begin{example}
\label{ex:fc_bk}
See \Cref{fig:bk_import_fc}.
\end{example}

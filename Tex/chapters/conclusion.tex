This thesis aims to answer the research question of whether the \acf{ape} can be applied to workflows in the data science field. To this end, \autoref{ch:ds_ontoloty} first limits the scope of potential use cases to \ac{eda}, predictive modeling, and text analysis, and based on that, models the domain ontology. Next, \autoref{ch:native_ape} elaborates on challenges and solutions while integrating the ontology into the \ac{ape} synthesis process. Additionally, it presents an alternative solving backend based on \acf{asp} that introduces heuristics and soft constraints to the workflow search. Finally, \autoref{ch:evaluation} evaluates the proposed approaches on the previously defined use cases and discusses how the lack of workflow-specific semantic information in the ontology affects them differently. It concludes with a comparison to the alternative backend and an overview of workflow construction experiments with generative \ac{ai}.

The defined ontology is modeled after the goal of producing general-purpose data science workflows using tools from standard Python libraries. However, this lack of semantic restrictions and the low-level nature of these operations result in the loss of type state dependencies in many tool sequences that seem essential for \ac{ape} to generate meaningful solutions. With exception to the \ac{eda} experiments, the produced Jupyter notebooks often contain non-executable steps or semantic errors, removing any statistical significance from the workflow artifacts. In contrast to the other domains \ac{ape} has been used in, the underlying types are not changing: Tables and columns are the data science ontology's primary types, and only a few tools modify them in their transition rules, further removing data dependencies. This affects feature engineering, text preprocessing, and modeling sequences the most, and thus, their constraint sets must include nearly all parameter assignments and tool orders. Creating these requires a deep technical understanding of the underlying tools and types, at which point, the user may prefer to compose the notebooks manually and benefit from the additional flexibility. However, the \ac{eda} experiments show how \ac{ape} can synthesize data science workflows with short and independent tool sequences. Any degree of freedom usually stems from varying column and style choices, which users may explore in the various generated solutions.

These problems mostly persist in the \ac{asp} backend as it uses the same core concept. However, the addition of soft constraints encoding domain, workflow, and user-specific solution preferences reintroduces some of the lacking semantic context and, hence, can improve the quality of the notebooks significantly without changing the search space. Furthermore, \ac{asp} enables the user to compose hard constraints directly interacting with the problem encodings.

The presented results are limited by the lack of specific application domains. Future versions of data science ontologies could increase contained semantic context by introducing a new dimension modeling the various states of workflows from the target domain. To counter the increased search complexity, a new version of \ac{ape} using \ac{asp} would tailor its encodings to the data science field by, e.g., including core concepts, such as tables and columns, in the workflow definition. Experiments with generative \ac{ai} have shown the potential for conversational interactions of users with constraint sets lowering the skill threshold for customizing \ac{asp} constraints without the limitations of templates.

This thesis lays the foundation for a prospective novel approach to automated construction of data science workflows with \ac{ape} and \ac{asp}. While challenges remain, the addition of \ac{asp} shows a promising path to improve the workflow quality and user experience. As generative \ac{ai} matures, it may significantly enhance semantic modeling and constraint composition, enabling the synthesis of more complex workflows while lowering the technical barriers for non-expert users.

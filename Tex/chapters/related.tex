Various ontology-based workflow construction approaches have been implemented in the past. However, most do not target workflows from the general data science domain. \ac{ape} \cite{kasalica2022synthesis}, the framework used in this thesis, is designed for scientific workflows and has been evaluated in multiple fields, including bioinformatics and geology. Some other solutions focus on data mining, a subfield of data science concerned with discovering structures in large datasets \cite{hand2007datamining}. For instance, \cite{OntoDMWorkflowComposition} transforms the users's input-output requirements into constraints to create a directed, acyclic workflow. Similar to APE, their ontology contains multiple taxonomies: knowledge and algorithms, which correspond to types and tools transitioning between those types.

Ontologies in data science exist. However, they often specialize in specific subfields, like the extensive OntoDM \cite{ontoDM} ontology for data mining, which includes relevant types, algorithms, their components, and tasks, or cover other scientific areas, such as the hundreds of ontologies found on the Ontology Lookup Service \cite{ols4} for bioinformatics. The IBM Data Science Ontology \cite{ibmdatascience} is one of the few general-purpose ontologies for data science. It is based on popular libraries, e.g., Pandas and scikit-learn, and includes concepts and annotations for both Python and R. The content is similar to the one used in this thesis but is not yet in a format fit for the automated construction of executable workflows. Other concepts for standardizing data formats and structures across services, such as the Microsoft Common Data Model \cite{microsoftcdm} or Data Catalog Vocabulary \cite{dcat}, lack the detail and context required for workflow construction.

Finally, many auto-ML concepts exist that aim to automate the entire data science process, including data preparation, feature engineering, and modeling \cite{he2021automl,hutter2019automated}. Some assist data scientists, while others target domain experts to develop \ac{ml} pipelines without technical knowledge. These pipelines, however, are often limited to a single task, the training and tuning of various model architectures, and thus, are not suitable for the general-purpose workflows targeted in this thesis.
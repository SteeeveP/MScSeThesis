In an era where data plays a key role in nearly all fields, the ability to compose workflows has never been more crucial. Automated solutions could not only speed up data analysis but also make it more accessible to non-data scientists or experts in other domains. However, the large number of tools and their complex interactions pose unique challenges to the effectiveness of automated workflow construction systems. The tool \ac{ape}\cite{kasalica2022synthesis} has successfully synthesized scientific workflows in various domains, such as geovisualization\cite{kasalica2019workflow} and bioinformatics\cite{kasalica2021ape}, with promising results. The synthesized workflows use types and tools from domain-specific ontologies not suitable for general-purpose data science tasks.

This thesis aims to answer the research question of whether \ac{ape} can be applied to generate executable workflows in the data science field. A success could enable less skilled data scientists to effectively receive drafts of required tool sequences and experts to more efficiently explore variants of their workflows using tools from the vast number of available libraries, all while interacting with the system on a semantic abstraction layer. To this end, the thesis needs to introduce a new tailored ontology and integrate it into the \ac{ape} synthesis process. Secondly, it needs to evaluate the resulting system on various data science use cases. Finally, it needs to compare the results to alternative approaches and discuss the implications of the ontology's design on the results.

The remainder of this thesis is structured as follows: \autoref{ch:background} will briefly overview the core concepts. Next, \autoref{ch:related} reviews related work in the field of automated workflow construction and data science ontologies. \autoref{ch:ds_ontoloty} limits the scope of potential use cases and models the domain ontology. \autoref{ch:native_ape} discusses the challenges of using the ontology with \ac{ape}, presents the transformation of solutions into executable Jupyter notebooks, and introduces an alternative solving backend based on \acf{asp}. \autoref{ch:evaluation} provides an empirical evaluation of the proposed approaches' effectiveness on the previously defined use cases, compares the performance of the \ac{asp} backend, and overviews results from experiments with generative \ac{ai}. Finally, \autoref{ch:conclusion} concludes the thesis with a summary of the key findings and an outlook on future research directions.

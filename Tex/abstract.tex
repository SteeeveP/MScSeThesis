
\thispagestyle{plain}
\begin{center}
\Large
% \textbf{\thetitle}

\vspace{10ex}

\large
% \textbf{\theauthor}

\vspace{5ex}
\textbf{Abstract}

\end{center}

In an era where data plays a key role in nearly all fields, the ability to compose workflows has never been more crucial. Automated solutions could not only speed up data analysis but also make it more accessible to non-data scientists or experts in other domains. The Automated Pipeline Explorer (APE) has successfully synthesized scientific workflows in various other domains, providing semantic abstraction to the technical tool layer. This thesis aims to answer the research question of whether APE can be applied to generate executable workflows in the data science field. To this end, it models a new domain ontology, integrates it into the APE synthesis process, and evaluates the resulting system. Furthermore, it also presents an alternative solving backend based on Answer Set Programming (ASP), extending the possibilities of APE, and peeks into the potential options of using generative artificial intelligence (AI) in workflow construction.

\vspace{5ex}

\begin{center}
\large \textbf{Zusammenfassung}
In einer Zeit, wo Daten eine primäre Rolle in nahezu allen Bereichen spielen, ist die Fähigkeit Workflows zu erstellen noch nie wichtiger gewesen. Automatisierte Lösungen könnten nicht nur die Datenanalyse beschleunigen, sondern diese auch zugänglicher zu Nicht-Data-Scientisten oder Experten anderer Bereiche machen. Das Tool Automated Pipeline Explorer (APE) hat bereits in einigen Domänen erfolgreich wissenschaftliche Datenverarbeitungs-Workflows erstellt und dabei eine semantische Abstraktionsebene zu den technischen Details der Tools geschaffen. Diese Arbeit versucht, die Forschungsfrage zu beantworten, ob APE auch für die Generierung von ausführbaren Workflows im Bereich Data Science verwendet werden kann. Dazu wird eine neue Domänenontologie erstellt, diese in den APE Syntheseprozess integriert und das resultierende System evaluiert. Außerdem wird ein alternativer Lösungsansatz basierend auf Answer Set Programming (ASP) vorgestellt, der die Möglichkeiten von APE erweitert, und einen Blick auf die potentiellen Ansätze der Verwendung von generativer Künstlicher Intelligenz (KI) in der Workflow-Konstruktion geworfen.
\end{center}
